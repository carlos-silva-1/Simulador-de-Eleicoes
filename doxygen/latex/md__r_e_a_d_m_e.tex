Primeiramente deve-\/se criar um servidor apache para que seja possível utilizar a aplicação. Todos os arquivos devem ser extraídos no diretório do servidor. Após esse passo, deve-\/se criar um banco de dados My\+SQL, e nele executar o arquivo \textquotesingle{}db.\+sql\textquotesingle{}. Esse projeto foi feito considerando um banco de dados de nome \textquotesingle{}urna\textquotesingle{}, o que pode ser modificado na linha 16 do arquivo \textquotesingle{}\mbox{\hyperlink{sql_8php}{sql.\+php}}\textquotesingle{}. Também nesse arquivo podem ser modificados o nome do servidor, o nome do usuário que pode acessar o servidor, e a senha do usuário.

A urna pode ser acessada pela página \textquotesingle{}index.\+html\textquotesingle{} no diretório raiz. Se o usuário desejar conferir os resultados das eleições, ele deve acessar \textquotesingle{}\mbox{\hyperlink{eleicoes_8php}{eleicoes.\+php}}\textquotesingle{}.

A documentação doxygen dos arquivos php estão na pasta \textquotesingle{}doxygen\textquotesingle{}, a documentação jsdoc dos arquivos javascript estão na pasta \textquotesingle{}js\textquotesingle{}, nas pastas \textquotesingle{}util jsdoc\textquotesingle{} e \textquotesingle{}script jsdoc\textquotesingle{}. 